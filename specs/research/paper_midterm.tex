\documentclass[twocolumn]{jarticle}
\usepackage[top=18truemm,bottom=18truemm,left=15truemm,right=15truemm]{geometry}
\usepackage[dvipdfmx]{graphicx,color}
\usepackage{authblk}
\usepackage{titlesec}
\usepackage{algpseudocode}
\usepackage{algorithm}
\usepackage{indentfirst}
\usepackage{url}
\usepackage{enumitem}
\usepackage[dvipdfmx]{graphicx}
\usepackage{here}

\usepackage{fancybox}
\usepackage{color}
\fancyput(-0.3in,0.5in){\color[rgb]{1,0,0}
{\fbox{\scalebox{1}{東京国際工科専門職大学学外秘}}}}


\pagestyle{empty}
\setlength{\columnsep}{7mm}

% タイトル(日本語+英語)を入れる
\title{大規模言語モデルを用いたUIの動的新規性制御:行動変容アプリにおける「飽き」の克服とエンゲージメント維持 \\
\sf Controlling Dynamic Novelty in UI Using Large Language Models: Overcoming "Boredom" and Maintaining Engagement in Behavior Change Apps}

\author{
% 著者名(日本語+ローマ字+学籍番号)を入れる
東京国際工科専門職大学 情報工学科 柏原 悠斗 Yuto KASHIWABARA (TK220307)
\\[1mm]
% 指導者名を入れる
(指導者:情報工学科 教授 武本 充治)
}

\affil{}
\date{}

\titleformat*{\section}{\normalsize\bfseries}
\titleformat*{\subsection}{\normalsize\bfseries}

\makeatletter
\def\@maketitle{% 
 \begin{center}% 
 {\Large\gt \@title \par}% タイトル 
 \end{center}% 
 \begin{flushright}% 
 {\normalsize \@author}% 著者 
 \end{flushright}%  
 \par\vskip 1.5em
 }
 \def\section{\@startsection {section}{1}{\z@}{-1.5ex plus -1ex minus -.2ex}{1.5 ex plus .2ex}{\normalsize\bf}}
\def\subsection{\@startsection {subsection}{1}{\z@}{-1.5ex plus -1ex minus -.2ex}{1.3 ex plus .2ex}{\small\bf}}
\def\subsubsection{\@startsection {subsubsection}{1}{\z@}{-1.5ex plus -1ex minus -.2ex}{.3 ex plus .2ex}{\normalsize \bf $\spadesuit$ }}
\makeatother

\begin{document}
\maketitle
\thispagestyle{empty}

% ここから本文
%================================================================%
\section{はじめに}
%================================================================%
\subsection{研究の背景}
日常生活における不安や心配は,多くの人々が経験する普遍的な課題である\cite{anxiety_prevalence_ref}.これらの心理的負荷は,しばしばタスクへの嫌悪感やストレスを高め,具体的な行動への着手を困難にし,先延ばし行動へと繋がることが指摘されている\cite{anxiety_procrastination_link_ref}.この問題に対する有効なアプローチとして,認知行動療法(CBT)に代表される心理学的技法が確立されており,特に複雑な課題を具体的な小タスクに分解する「行動活性化」は,着手への抵抗感を低減させる上で効果的であることが知られている\cite{behavioral_activation_ref}.

こうした背景から,近年では認知行動療法を手軽に実践できるデジタルメンタルヘルス(DMH)アプリケーションが数多く開発され,広く普及している\cite{dmh_market_growth_ref}.しかし,その有効性には大きな障壁が存在する.多くのDMHアプリは,利用開始から数週間以内にユーザが離脱してしまうという深刻なアドヒアランス(継続利用)の問題を抱えているのである\cite{dmh_adherence_problem_ref}.この一因として,静的なユーザインターフェース(UI)や画一的な通知が,ユーザの「慣れ(習慣化)」を引き起こし,次第に注意を引かなくなってしまう点が挙げられる\cite{habituation_ui_blindness_ref}.この課題に対し,個人の状況に応じて介入のタイミングや内容を最適化するジャストインタイム適応介入(JITAI)が有望なアプローチとされているが,介入が最適なタイミングで提示されたとしても,ユーザがそれを受け入れる心理状態にあるか,すなわち「介入受容性(Receptivity)」が担保されなければ効果は限定的であることもまた,重要な課題として認識されている\cite{jitai_receptivity_challenge_ref}.

\subsection{課題設定}
以上の背景を踏まえ,本研究では介入受容性の低下,すなわち習慣化の問題は,介入の内容やタイミングだけでなく,その提示方法に起因する部分が大きいのではないかと考える.静的で予測可能なUIが習慣化を生むのであれば,逆にUIの新規性(Novelty)を文脈に応じて動的に制御することにより,利用者の注意を喚起し,重要な介入への気づきを促すことができる可能性がある.すなわち,介入の重要度や緊急性,そして利用者の状況に応じてUIの見た目や振る舞いを変化させることで,習慣化を防ぎ,介入受容性を高めることができると推測される.

%そこで本稿では,この推測を検証するための第一歩として,大規模言語モデル(LLM)を用いてUIの新規性を動的に制御する,心配事管理アプリケーションのプロトタイプを設計・実装する.LLMが持つ柔軟な生成能力を活用しつつも,その確率的な性質に起因する安全性や再現性の課題を,堅牢なドメイン固有言語(DSL)を介するアーキテクチャによって克服することを目指す.その上で,開発したプロトタイプを用いた短期ユーザ評価を通じて,提案手法がユーザの行動や認識にどのような影響を与えうるか,その可能性を示す定性的な示唆を得ることを目標とする.
この「習慣化による介入受容性の低下」という根深い課題に対し,取りうる対策は多岐にわたる.例えば,ゲーミフィケーション要素を適応的に変化させる,あるいは提示するコンテンツそのものを多様化させるといったアプローチが考えられる.しかし,それらの手法もまたパターン化し,新たな習慣化を生む可能性がある.そこで本研究では,より根源的な対策として,ユーザが必ず接する情報提示の「器」であるユーザインターフェース(UI)自体の新規性(Novelty)を,文脈に応じて動的に制御するという手法に着目する.このアプローチを実現する上で,近年の大規模言語モデル(LLM)が持つ高度な文脈理解能力と柔軟な生成能力は,従来技術では困難であった,きめ細やかで無限に近いUI表現を可能にする技術的基盤となりうる.

一方で,LLMの確率的な性質は,UI生成の品質や再現性,セキュリティに重大な懸念をもたらす.そこで本稿では,この推測を検証する第一歩として,LLMの能力を安全かつ堅牢に活用するドメイン固有言語(DSL)を介したアーキテクチャを提案し,それに基づいた心配事管理アプリケーションのプロトタイプを設計・実装する.そして,開発したプロトタイプを用いた短期ユーザ評価を通じて,提案手法がユーザの行動や認識に与える影響について,その可能性を示す定性的な示唆を得ることを目標とする.


%================================================================%
\section{関連研究と本研究の位置付け}\label{seciton-positioning}
%================================================================%
本研究は,(A) ジャストインタイム適応介入,(B) LLMによるUI生成,(C) 説得的技術とデジタルメンタルヘルスの3つの研究領域が交差する点に位置付けられる.

% \paragraph{(A) JITAIにおける介入受容性と習慣化の課題}
% JITAIは,個人の状態や文脈に応じて最適なタイミングで支援を提供するアプローチとして注目されている\cite{nahum2016jitai}.しかし,その効果を最大化するためには,ユーザが介入を受け入れる心理状態にあるか,すなわち「介入受容性(Receptivity)」の確保が不可欠である\cite{mohammadi2022detecting}.多くのJITAI研究は介入の「内容」と「タイミング」の最適化に焦点を当ててきたが,介入の「提示方法」が静的であることに起因する課題は十分に議論されてこなかった.同じ形式の通知やUIが繰り返されると,ユーザは次第にそれに慣れ,注意を払わなくなる「習慣化(Habituation)」が生じ,介入効果が減衰することが指摘されている.実際,介入頻度を高めることが短期的には行動を促すものの,長期的にはアドヒアランスの低下を招く可能性が示唆されており,これは介入疲労(Intervention Fatigue)や習慣化が一因であると考えられる\cite{rabbi2015behavioral}.本研究は,この習慣化による受容性の低下というJITAIの根本課題に対し,介入の提示方法,すなわちUIの動的な新規性制御という新しいアプローチで挑むものである.

% \paragraph{(B) LLMによるUI生成の品質・安全性という課題}
% LLMを用いてUIを自動生成する研究は活発化しているが,その多くはUIコードを直接生成するアプローチを採用しており,実用化には深刻な課題が残る.第一に,LLMは構文的に正しく,かつ視覚的に意図通りのUIコードを一貫して生成することに困難を抱えており,品質と再現性が保証されない\cite{wu2024uicoder}.第二に,より重大な問題として,
% 生成されたコードにセキュリティ脆弱性が含まれる危険性がある.例えば,GitHub CopilotのようなLLMベースのコード生成アシスタントが生成したコードの約30\%に,コードインジェクションなどの脆弱性が含まれていたとの報告がある\cite{asare2023large}.本研究で採用する「LLM-Hardened DSL」アーキテクチャは,LLMにUIコードを直接生成させるのではなく,安全性が検証されたDSLのみを生成させることで,これらの品質,再現性,そしてセキュリティ上の課題を原理的に回避するアプローチである.

% \paragraph{(C) 説得的技術における新規性とユーザビリティのトレードオフ}
% 行動変容を促す説得的技術において,UIの新規性(Novelty)は,ユーザの関心を引きつけ,エンゲージメントを高める「新規性効果(Novelty Effect)」をもたらす重要な要素である\cite{fogg2002persuasive}.しかし,無秩序なUIの変化は,ユーザに混乱をもたらし,操作方法の学習を妨げる.これは,適応型UI(AUI)の研究分野で指摘されているように,UIの一貫性を損なうことで認知負荷を高め,ユーザビリティを著しく低下させる危険性をはらむ\cite{gong2022adaptive}.すなわち,エンゲージメント向上と認知負荷低減の間には「新規性-ユーザビリティのトレードオフ」が存在する.したがって,単に動的なUIを導入するだけでなく,その変化の度合い(新規性のレベル)をユーザの文脈に応じていかにインテリジェントに「制御」するかが極めて重要となる.

\paragraph{(A) JITAIにおける介入受容性と習慣化の課題}
多くのJITAI研究は介入の内容とタイミングに注力してきたが\cite{nahum2016jitai},静的な提示方法が引き起こすユーザの習慣化(Habituation)と,それに伴う介入受容性(Receptivity)の低下という課題は未解決であった\cite{mohammadi2022detecting}\cite{rabbi2015behavioral}.本研究は,この習慣化の問題に対し,UIの新規性を動的に制御することで介入効果の減衰を防ぐという,新たなアプローチで挑むものである.

\paragraph{(B) LLMによるUI生成の品質・安全性という課題}
LLMによるUIコードの直接生成は,品質・再現性の欠如\cite{wu2024uicoder}やセキュリティ脆弱性\cite{asare2023large}といった深刻な実用上の課題を抱える.本研究ではこの問題を回避するため,LLMにUIコードではなく安全性が検証されたDSLによるUIレイアウトを生成させるアーキテクチャを採用する.これにより,LLMの柔軟な生成能力を,品質と安全性を担保した上で活用する.

\paragraph{(C) 説得的技術における新規性とユーザビリティのトレードオフ}
UIの新規性(Novelty)はエンゲージメントを高めるが\cite{fogg2002persuasive},過度で無秩序な変化はユーザビリティを損ない,ユーザの認知負荷を高める\cite{gong2022adaptive}.この「新規性-ユーザビリティのトレードオフ」の存在を踏まえ,本研究では単にUIを動的にするだけでなく,その新規性のレベルをユーザの文脈に応じてインテリジェントに制御することを重要な要件とする.

\vspace{2mm}
\noindent
以上の背景を踏まえ,本研究は以下の3点において独自性を持つ試みである.
\begin{enumerate}[label=(\Alph*)] 
    \item \textbf{介入受容性への新しいアプローチの実装:} JITAIにおける習慣化と受容性低下という課題に対し,UIの新規性を文脈に応じて動的に制御するという新しい介入手法を実装し,そのプロトタイプを開発する.
    \item \textbf{安全なLLM駆動UIアーキテクチャの実証:} LLMによるUIコード直接生成の品質・セキュリティリスクを回避するため,LLM-Hardened DSLの思想に基づき,LLMの能力を安全かつ再現可能な形でヘルスケア応用に接続するアーキテクチャを設計・実装する.
    \item \textbf{新規性トレードオフの定性的評価:} 新規性-ユーザビリティのトレードオフに対し,新規性レベルを動的に調整する手法が,ユーザの受容性や認知負荷にどのような影響を与えうるか,短期ユーザ評価を通じて予備的な示唆を得る.
\end{enumerate}

%================================================================%
\section{研究目標}
%================================================================%
本研究の最終的なゴールは,LLMを用いた動的UIが,心配事を抱えるユーザの着手行動と不安感に与える影響を定量的に評価することにある.以下の4点を達成可能な目標として設定する.

\begin{enumerate}
    \item \textbf{プロトタイプの実装:} LLM-Hardened DSLアーキテクチャに基づき,UIの新規性を制御可能な心配事管理アプリのプロトタイプを実装する.これには,比較対象となる固定UIバージョンも含む.(A)
    \item \textbf{技術的実現可能性の評価:} 提案アーキテクチャの性能(UI生成速度)と再現性を定量的に測定し,実用性を評価する.(B)
    \item \textbf{短期ユーザ評価の実施:} 開発したプロトタイプを少数(5名程度)の被験者に短期間(1週間程度)利用してもらい,利用ログの収集とインタビューを実施する.(B)
    \item \textbf{定性的示唆の獲得:} 収集した定量的・定性的データから,提案する動的UIがユーザの行動や認識(受容性,認知負荷など)に与える影響に関する予備的な知見(示唆)を抽出する.(C)
\end{enumerate}

%================================================================%
\section{提案手法}
%================================================================%
\subsection{LLM-Hardened DSLアーキテクチャ}
% サーバサイドLLM(Gemini 2.5 Flash-Lite)が,クライアントから送られる状況スナップショットと候補データを入力として,UI DSLのJSONを生成する.このDSLは\texttt{headline},\texttt{cards},\texttt{widget}など,
% 事前に定義され安全性が検証されたホワイトリスト型の要素のみで構成される.新規性(Novelty)レベルは\texttt{low|med|high}で指定でき,
% を指摘でき,色,配置,軽微なアニメーションの強度を制御する.LLMの生成は決定論的モード(temperature, topP固定)で行い,出力はスキーマ検証を通過したもののみ採用することで,再現性と安全性を確保する.
サーバサイドLLMが,クライアントから送信される状況データを入力として,UIレイアウトをDSLで出力する.UIはホワイトリスト型の要素のみで構成される.この際,新規性(Novelty)レベルが指定でき,色,配置,アニメーションの有無などを制御する.

\subsection{心配事管理アプリ}
アプリは,心配事の登録と,\textbf{実態を掴む}→\textbf{方針を立てる}→\textbf{細かく分割する}という認知行動療法(CBT)の技法に基づいたガイドを提供する.心配事に紐づく具体的なタスクを小さく分割し,常に「2分から始められる行動」を提示することで着手の障壁を下げる.なお,iOSアプリとして開発する.
%データは原則として端末に保存し,UI生成に必要な情報のみを匿名化された要約としてサーバへ送信する.

\subsection{動的UIの作用機序}
状況情報(時刻,予定の空き,位置カテゴリ等)とアプリ内データ(心配事,タスク,進捗)から優先スコアを計算し,提示すべきタスク候補を選定する.このタスクを提示するためのUIを,文脈に応じた新規性レベルでLLMが生成する.
%例えば,長期間着手されていない重要タスクがあり,かつユーザに時間的余裕がある状況では,新規性レベルを上げ,注意を引きつけるUIを生成する.
この仕組みにより,いま取り組むべき行動が,最も受容されやすい形でユーザの前に現れることを目指す.

%================================================================%
\section{研究計画}
%================================================================%
%提案手法を実装するため,クライアントはSwiftUIシェルと\texttt{WKWebView}で構成し,サーバはBun/Honoで構築する.UIはホワイトリスト化されたコンポーネントのみで構成し,個人情報はサーバに保存しないなど,安全性とプライバシーに配慮する.研究の科学的信頼性を担保するため,設定のバージョン管理,LLM呼び出しのパラメータ固定,リプレイAPIによる事後検証など,再現性確保に重点を置く.

%\paragraph{アーキテクチャ} クライアントはSwiftUIシェルと\texttt{WKWebView}で構成する.Web側に軽量なレンダラを実装し,サーバから受信したUI~DSLを解釈して描画する.サーバはBun/Hono,\texttt{/v1/config}(設定配布),\texttt{/v1/ui/generate}(UI生成),\texttt{/v1/score/rank}(優先付け),\texttt{/v1/events/batch}(イベント収集),\texttt{/v1/replay/generate}(リプレイ)といったAPIを提供する.この構成は,LLM-Hardened DSLの思想を具現化するものである.

%\paragraph{再現性} 研究の科学的信頼性を担保するため,再現性確保に重点を置く.設定は\texttt{configVersion=v1}として凍結し,レスポンスに設定スナップショットを同梱する.LLM呼出は温度・topP・topK固定,seed指定戦略を採用する.さらに,一度生成したDSLはキャッシュし,障害発生時やデバッグ時には直近のキャッシュまたは固定テンプレートにフォールバックする.全イベントログには,どのUI生成レスポンスに基づいた操作かを追跡可能なIDを付与し,リプレイAPIによる事後検証を可能にする.

%\paragraph{優先スコア} スコアは,importance, urgency, relief, deadlineProximity, contextFit, timeFit, staleness, energyMatch, switchCostなどの要素の線形結合で算出する.重みは研究期間中はプリセット値に固定し,過度なパーソナライゼーションによる操作(manipulation)のリスクを避け,透明性と再現性を優先する.これは説得的技術を倫理的に応用するための設計上の選択である.

本研究は以下の3ヶ月間のスケジュールで遂行する.各作業の依存関係を図\ref{fig:placeholder}に示す.
\begin{figure}[H]
    \centering
    \includegraphics[width=1\linewidth]{resources/pipeline-schedule.png}
    \caption{研究タスクの依存関係図}
    \label{fig:placeholder}
\end{figure}
%\paragraph{9月:MVPアプリ開発}
%ユーザ検証に必要なコア機能を持つプロトタイプを完成させる.クライアントに心配事登録機能,サーバサイドに動的UI生成機能を実装する.その後,ユーザが本アプリを利用可能な水準になるよう,LLMが扱うプロンプトとDSLを調整する.また,この時点で実験手順を作成する.
%\begin{itemize}
%    \item Week 1: 基盤設計(DSL v1.1仕様確定,サーバー/クライアント骨格構築)
%    \item Week 2-3: コア機能実装
%    \item Week 4: LLM連携とプロトタイプ完成
%\end{itemize}

%\paragraph{10月:技術的基盤の確立と実験準備}
%アプリのデバッグを行い提案アーキテクチャの堅牢性を検証し,比較実験の準備を完了させる.また,被験者募集も同時に行う.
%\begin{itemize}
%    \item Week 5-6: 堅牢性検証(再現性テスト,性能測定,フォールバック機構実装)
%    \item Week 7-8: 比較実験準備(固定UI版開発,評価指標確定,被験者リクルート)
%\end{itemize}

%\paragraph{11月:ユーザ検証と論文執筆}
%プロトタイプによる実験から示唆を得て,論文を完成させる.
%\begin{itemize}
%    \item Week 9: ユーザ検証の実施(アプリ利用)
%    \item Week 10: データ分析(ログ分析,インタビュー)と示唆の抽出
%    \item Week 11-12: 卒業論文執筆と発表準備
%\end{itemize}

%本研究のスケジュールは,アプリ開発と実験計画という2つの主要なパイプラインが並行して進み,最終的に実験フェーズで統合される構成となっている.

まず,研究全体の基盤となる要件定義・設計を行う.この設計に基づき,アプリの核となるクライアント(心配事登録機能)とサーバサイド(動的UI生成機能)の実装を並行して進める.両方の実装が完了し,コア部分の実装が定まった段階で,初めてLLMが解釈するプロンプトとDSLの調整が可能になる.このプロトタイプの堅牢性検証(デバッグと動作確認)を行い,iOS上において提案アーキテクチャの実現可能性を示す.

同時に,要件定義を反映した実験設計を行い,倫理審査を経て被験者募集を進める.

最終的に,完成したプロトタイプと被験者が揃うことで実験が可能となる.この実験において,本アーキテクチャにおけるユーザの受容性を定性的に明らかにし,得られた知見を分析・考察し,卒論執筆をもって本研究を締めくくる.

\subsection{進捗状況}
現在,類似研究調査ならびにアプリMVPの要件定義が完了している.次の段階でアプリの設計を行い,実装計画の具体化を進めつつ,実験のための書類を作成する.

\section{おわりに}\label{sec:conc}
%本研究は,デジタルメンタルヘルス(DMH)領域におけるユーザの「習慣化」とそれに伴う「介入受容性の低下」という根深い課題に対し,大規模言語モデル(LLM)を用いてユーザインターフェース(UI)の新規性を動的に制御するという新たなアプローチを提案し,その実現可能性と潜在的効果を検証するものである.そのために,LLMの柔軟な生成能力を安全なドメイン固有言語(DSL)を介して活用する「LLM-Hardened DSL」アーキテクチャを設計し,それに基づいた心配事管理アプリケーションのプロトタイプを開発する計画である.そして,短期的なユーザ評価を通じて,提案手法がユーザの注意喚起やタスクへの意識にどのような影響を与えうるか,その可能性を示す定性的な示唆を得ることを目指す.本研究は,JITAIにおける介入の「提示方法」という新たな次元に光を当て,その動的最適化の重要性を実践的に探求する第一歩となる.

%本研究は,提案手法の技術的な実現可能性を検証し,そのコンセプトがユーザに与える影響の仮説を立てるための探索的な第一歩と位置づけている.そのため,時間的制約からユーザ評価は少数名による短期的な定性評価に留める計画である.提案手法が長期的なアドヒアランスや実際の不安軽減に寄与するかを明らかにするためには,本研究で得られた知見を基に,より多くの被験者を対象とした長期間のランダム化比較試験(RCT)による定量的な評価が今後の課題となる.また,UIの動的変化とユーザの認知負荷との間に存在する「新規性-ユーザビリティのトレードオフ」を解明し,個々のユーザ特性に応じた最適な制御方法を見出すことも重要な研究テーマとなろう.

%本研究で提案するコンセプトとアーキテクチャは,広範な領域への応用が期待される.メンタルヘルス領域に留まらず,禁煙や運動習慣といった他の健康行動変容,あるいは適応的学習システムといった教育分野など,ユーザの持続的なエンゲージメントが求められる様々な応用において,新たな道を拓く可能性を秘めている.この挑戦を通じて得られる知見が,よりパーソナライズされ,人間中心で,真に効果的なデジタル介入の未来に貢献することを期待する.

%本稿で計画する研究は,LLMという先進技術を,人間の行動変容という機微な領域へ安全かつ効果的に応用するための具体的な道筋の一つを示す試みである.この挑戦を通じて得られる知見が,よりパーソナライズされ,人間中心で,真に効果的なデジタル介入の未来に貢献することを期待する.

本研究は,デジタルヘルスアプリにおける「習慣化」という深刻な課題に対し,本研究はUIの動的新規性制御という解決策を提示する.LLMとDSLを組み合わせた安全なアーキテクチャを実装・評価し,介入効果を持続させるための新たな設計指針を得ることを成果とする.将来的には,本研究の知見が,より人間中心的で効果的なデジタル介入技術の基盤となり,様々な行動変容・学習支援分野へ貢献することが期待される.

\section*{安全・審査・プライバシー} 本研究は被験者の機微な情報を取り扱うために,実験実施までに,東京国際工科専門職大学の倫理審査委員会の承認を10月に得る予定である.
% UIはホワイトリスト化されたコンポーネントのみで構成し,予期せぬUIや危険なアクションが生成されることを防ぐ.個人情報や精密な位置情報はサーバに保存しない.生体情報はユーザの明示的な許諾があった場合のみ利用する.
%心拍/HRVの粗い区分値(例:安静時,高活動時)にオンデバイスで要約してから利用する.通信はTLSで暗号化し,端末上のデータはOSが提供する保護領域に保存する.

%--- 参考文献にBibTeXを使う場合はこちら
\bibliographystyle{tipsj}
\bibliography{myref}
%---------------------------------

\end{document}
